\documentclass{article}
\usepackage[utf8]{inputenc}
\usepackage{amsmath}
\usepackage{graphicx}

\title{Uso de GitHub Issues para Gestión de Tareas y Seguimiento de Problemas}
\author{}
\date{}

\begin{document}

\maketitle

\section{Introducción}
GitHub \textbf{Issues} es una herramienta poderosa para gestionar tareas, reportar bugs, sugerir nuevas características o mejoras, y realizar un seguimiento del progreso dentro de un proyecto de software. Te permite colaborar y organizar el trabajo de manera eficiente dentro de un repositorio.

\section{1. Acceder a GitHub Issues}
Para utilizar GitHub Issues en un proyecto, sigue estos pasos:
\begin{itemize}
    \item Abre el repositorio en GitHub donde quieres reportar o gestionar issues.
    \item Haz clic en la pestaña \textbf{"Issues"} en la parte superior de la página.
\end{itemize}

Si el repositorio no tiene la pestaña \textbf{Issues} activada, los propietarios pueden habilitarlo en \textbf{Settings $\rightarrow$ Features $\rightarrow$ Issues}.

\section{2. Crear un Issue Nuevo}
\begin{enumerate}
    \item En la pestaña \textbf{Issues}, haz clic en el botón \textbf{"New Issue"}.
    \item Ingresa un título descriptivo. Este debería resumir el problema o tarea.
    \item En la caja de descripción, escribe los detalles del issue:
    \begin{itemize}
        \item Descripción clara del problema.
        \item Pasos para reproducir un bug (si aplica).
        \item Soluciones propuestas o ideas para mejora.
        \item Capturas de pantalla o links relevantes, si son útiles.
    \end{itemize}
    \item (Opcional) Agrega etiquetas (\textbf{Labels}) para categorizar el issue, como:
    \begin{itemize}
        \item \textbf{bug}: Si es un error o problema.
        \item \textbf{enhancement}: Para nuevas características o mejoras.
        \item \textbf{documentation}: Para cambios en la documentación.
    \end{itemize}
    \item (Opcional) Asigna el issue a un colaborador específico o a ti mismo si eres quien va a trabajar en la tarea.
    \item Haz clic en \textbf{"Submit new issue"} para crear el issue.
\end{enumerate}

\section{3. Seguir el Progreso de un Issue}
\subsection{a. Comentarios}
Cualquiera que tenga acceso al repositorio puede dejar comentarios en un issue. Esto permite:
\begin{itemize}
    \item \textbf{Discutir} posibles soluciones o pasos adicionales.
    \item \textbf{Resolver dudas} y compartir ideas.
\end{itemize}
En cada comentario se puede mencionar a un colaborador usando \texttt{@username} para notificar a esa persona y pedir su intervención.

\subsection{b. Cerrar un Issue}
Cuando un issue se resuelve (ya sea que se haya corregido un bug o completado una tarea), el creador o cualquier colaborador con permisos puede cerrarlo:
\begin{itemize}
    \item Haz clic en el botón \textbf{"Close issue"} en la parte inferior del issue.
\end{itemize}
Los \textbf{commits} que mencionen el número del issue en su mensaje de commit, como en \texttt{Fixes \#123}, pueden cerrar automáticamente el issue relacionado cuando el commit se fusione en la rama principal.

\subsection{c. Referenciar Commits y Pull Requests}
Puedes vincular un issue a un commit o pull request para rastrear cómo el código resuelve el problema. En el mensaje del commit o PR, puedes usar \texttt{\#} seguido del número del issue, por ejemplo:
\begin{verbatim}
git commit -m "Soluciona el bug en la función de login. Fixes #45"
\end{verbatim}
Esto vinculará el commit o PR con el issue mencionado.

\section{4. Uso Avanzado de Issues}
\subsection{a. Milestones (Hitos)}
Puedes agrupar issues bajo \textbf{milestones} o hitos, que son metas más grandes del proyecto. Esto permite dividir el trabajo en etapas o lanzamientos.
\begin{itemize}
    \item En la pestaña de \textbf{Milestones}, crea un nuevo milestone.
    \item Asigna un conjunto de issues al milestone para hacer un seguimiento de qué tareas deben completarse para cumplir un objetivo.
\end{itemize}

\subsection{b. Projects (Tableros de Proyecto)}
GitHub permite organizar issues en tableros estilo Kanban a través de \textbf{Projects}. Los proyectos permiten visualizar el estado de las tareas mediante tarjetas que se mueven por columnas (como "To Do", "In Progress", "Done").
\begin{itemize}
    \item Crea un nuevo proyecto en la pestaña \textbf{Projects}.
    \item Agrega issues como tarjetas al proyecto y organízalos en columnas.
    \item A medida que trabajas en un issue, mueve la tarjeta de columna en columna para indicar su progreso.
\end{itemize}

\subsection{c. Etiquetas Personalizadas (Custom Labels)}
Además de las etiquetas predefinidas, puedes crear etiquetas personalizadas para gestionar el flujo de trabajo. Por ejemplo:
\begin{itemize}
    \item \textbf{critical}: Para problemas que requieren atención inmediata.
    \item \textbf{wontfix}: Si decides que un problema no será abordado.
\end{itemize}

\section{5. Mejores Prácticas para Usar Issues}
\begin{itemize}
    \item \textbf{Usa títulos descriptivos}: Un título claro hace que los issues sean fáciles de identificar y priorizar.
    \item \textbf{Divide las tareas}: Si una tarea es grande, divídela en varios issues pequeños.
    \item \textbf{Usa etiquetas consistentemente}: Mantén un sistema de etiquetas claro para facilitar el filtro y la priorización.
    \item \textbf{Cierra issues innecesarios}: Si un issue ya no es relevante, ciérralo para mantener el flujo de trabajo limpio.
\end{itemize}

\section{Conclusión}
Los \textbf{GitHub Issues} son una herramienta útil para gestionar tareas y hacer seguimiento a problemas dentro de un proyecto. Puedes crear issues, asignar colaboradores, agregar etiquetas, y referenciar issues desde commits o pull requests. \textbf{Milestones} y \textbf{Projects} permiten organizar el trabajo de forma más estructurada.

\end{document}

